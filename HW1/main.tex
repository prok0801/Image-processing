\documentclass{article}

\newif\ifanswers
\answerstrue % comment out to hide answers

\usepackage[compact]{titlesec}
\usepackage{fancyhdr} % Required for custom headers
\usepackage{lastpage} % Required to determine the last page for the footer
\usepackage{extramarks} % Required for headers and footers
\usepackage[usenames,dvipsnames]{color} % Required for custom colors
\usepackage{graphicx} % Required to insert images
\usepackage{listings} % Required for insertion of code
\usepackage{courier} % Required for the courier font
\usepackage{lipsum} % Used for inserting dummy 'Lorem ipsum' text into the template
\usepackage{enumerate}
\usepackage{enumitem}
\usepackage{subfigure}
\usepackage{booktabs}
\usepackage{amsmath, amsthm, amssymb}
\usepackage[maxbibnames=99,maxcitenames=1]{biblatex}
\usepackage{caption}
\usepackage{hyperref}
\captionsetup[table]{skip=4pt}
\usepackage{framed}
\usepackage{bm}
\usepackage{minted}
\usepackage{soul}
\usepackage[utf8]{vietnam}
\usepackage[vietnamese,english]{babel}


\addbibresource{references.bib} %Import the bibliography file
\AtNextBibliography{\small}

\usepackage{tikz}
\usetikzlibrary{positioning, patterns, fit}


% Margins
\topmargin=-0.45in
\evensidemargin=0in
\oddsidemargin=0in
\textwidth=6.5in
\textheight=9.0in
\headsep=0.25in

\linespread{1.1} % Line spacing

% Set up the header and footer
\pagestyle{fancy}
\rhead{\hmwkAuthorName} % Top left header
\lhead{\hmwkClass: \hmwkTitle} % Top center head
\lfoot{\lastxmark} % Bottom left footer
\cfoot{} % Bottom center footer
\rfoot{Page\ \thepage\ of\ \protect\pageref{LastPage}} % Bottom right footer
\renewcommand\headrulewidth{0.4pt} % Size of the header rule
\renewcommand\footrulewidth{0.4pt} % Size of the footer rule

\setlength\parindent{0pt} % Removes all indentation from paragraphs

\newenvironment{answer}{
    % Uncomment this if using the template to write out your solutions.
    {\bf Answer:} \sf \begingroup\color{red}
}{\endgroup}%
%----------------------------------------------------------------------------------------
%	CODE INCLUSION CONFIGURATION
%----------------------------------------------------------------------------------------

\definecolor{MyDarkGreen}{rgb}{0.0,0.4,0.0} % This is the color used for comments
\definecolor{shadecolor}{gray}{0.9}

\lstloadlanguages{Python} % Load Perl syntax for listings, for a list of other languages supported see: ftp://ftp.tex.ac.uk/tex-archive/macros/latex/contrib/listings/listings.pdf
\lstset{language=Python, % Use Perl in this example
        frame=single, % Single frame around code
        basicstyle=\footnotesize\ttfamily, % Use small true type font
        keywordstyle=[1]\color{Blue}\bf, % Perl functions bold and blue
        keywordstyle=[2]\color{Purple}, % Perl function arguments purple
        keywordstyle=[3]\color{Blue}\underbar, % Custom functions underlined and blue
        identifierstyle=, % Nothing special about identifiers
        commentstyle=\usefont{T1}{pcr}{m}{sl}\color{MyDarkGreen}\small, % Comments small dark green courier font
        stringstyle=\color{Purple}, % Strings are purple
        showstringspaces=false, % Don't put marks in string spaces
        tabsize=5, % 5 spaces per tab
        %
        % Put standard Perl functions not included in the default language here
        morekeywords={rand},
        %
        % Put Perl function parameters here
        morekeywords=[2]{on, off, interp},
        %
        % Put user-defined functions here
        morekeywords=[3]{test},
       	%
        morecomment=[l][\color{Blue}]{...}, % Line continuation (...) like blue comment
        numbers=left, % Line numbers on left
        firstnumber=1, % Line numbers start with line 1
        numberstyle=\tiny\color{Blue}, % Line numbers are blue and small
        stepnumber=5 % Line numbers go in steps of 5
}

% Creates a new command to include a perl script, the first parameter is the filename of the script (without .pl), the second parameter is the caption
\newcommand{\perlscript}[2]{
\begin{itemize}
\item[]\lstinputlisting[caption=#2,label=#1]{#1.pl}
\end{itemize}
}

%----------------------------------------------------------------------------------------
%	NAME AND CLASS SECTION
%----------------------------------------------------------------------------------------

\newcommand{\hmwkTitle}{Homeworks} % Assignment title
\newcommand{\hmwkClass}{INT3404E 20 - Image Processing} % Course/class
\newcommand{\hmwkAuthorName}{Trần Phương Linh} % Your name

\newcommand{\ifans}[1]{\ifanswers \color{red} \vspace{5mm} \textbf{Solution: } #1 \color{black} \vspace{5mm} \fi}

% Chris' notes
\definecolor{CMpurple}{rgb}{0.6,0.18,0.64}
\newcommand\cm[1]{\textcolor{CMpurple}{\small\textsf{\bfseries CM\@: #1}}}
\newcommand\cmm[1]{\marginpar{\small\raggedright\textcolor{CMpurple}{\textsf{\bfseries CM\@: #1}}}}

%----------------------------------------------------------------------------------------
%	TITLE PAGE
%----------------------------------------------------------------------------------------
\title{
\vspace{-1in}
\textmd{\textbf{\hmwkClass:\ \hmwkTitle} \\ \hmwkAuthorName}\\
}
\author{}
% \date{\textit{\small Updated \today\ at \currenttime}} % Insert date here if you want it to appear below your name
\date{}

\setcounter{section}{0} % one-indexing
\begin{document}
\maketitle


\section{The result of the function grayscale\_image}

\subsection{Source code}
\begin{lstlisting}[caption={Code of grayscale\_image() function}, label={grayscale\_image}]
def grayscale_image(image):
    height, width = image.shape[:2]
    img_gray = np.zeros((height, width), dtype=np.uint8)
    for i in range(height):
        for j in range(width):
    
            B = image[i, j, 0]
            G = image[i, j, 1]
            R = image[i, j, 2]
            
            gray_value = 0.299 * R + 0.587 * G + 0.114 * B

            img_gray[i, j] = gray_value

    return img_gray
\end{lstlisting}

\subsection{Input}
\begin{itemize}
    \item Input: 
    \begin{itemize}
        \item \lstinline{image}: a Numpy array containing the image data
    \end{itemize}
    \item Algorithm:
    \begin{itemize}
        \item Get the dimension of the image
        \item Create a Numpy array with the same dimension as the image this will be used to stored the new grayscale image
        \item Convert each pixel of an original image to a grayscale image using the following formula for each pixel:
        \[p = 0.299 * R + 0.587 * G + 0.114 * B \]
    \end{itemize}
\end{itemize}
\subsection{Output}
\begin{itemize}
    \item Return the converted image
\end{itemize}
\begin{figure}[H]
    \centering
    \includegraphics[width=0.75\textwidth]
    {lena_gray.jpg}
    \caption{Grayscale Image}
    \label{fig:grayscale_img}
\end{figure}

\section{The result of the function flip\_image}
\subsection{Source code}
\begin{lstlisting}[caption={Code of flip\_image() function}, label={flip\_image}]
def flip_image(image):
    return cv2.flip(image, 1)
\end{lstlisting}
\subsection{Input}
\begin{itemize}
    \item Input:
    \begin{itemize}
        \item \lstinline{image}: A Numpy array containing the image data
    \end{itemize}
    
    \item Algorithm:
    \begin{itemize}
        \item Use function : \lstinline{cv2.flip(image, 1)} 
        \begin{itemize}
            \item \lstinline{image} The data of the image
            \item \lstinline{flipCode = 1} Specify that the image will be flipped along the y-axis
        \end{itemize}
    
    \end{itemize}
\end{itemize}

\subsection{Output}
\begin{itemize}
    \item Return the flipped image:
\end{itemize}

\begin{figure}[H]
    \centering
    \includegraphics[width=0.75\textwidth]{lena_gray_flipped.jpg}
    \caption{Flipped Grayscale Image}
    \label{fig:flipped_grayscale_img}
\end{figure}


\section{The result of the function rotate\_image}
\subsection{Source code}
\begin{lstlisting}[caption={Code of rotate\_image() function}, label={rotate\_image}]
def rotate_image(image, angle):
    height, width = image.shape[:2]
    rotation_matrix = cv2.getRotationMatrix2D((width / 2, height / 2), angle, 1)
    rotated_image = cv2.warpAffine(image, rotation_matrix, (width, height))
    return rotated_image
\end{lstlisting}

\subsection{Input}

\begin{itemize}
    \item Input:
    \begin{itemize}
        \item \lstinline{image}: A Numpy array containing the image data
        \item \lstinline{angle}: The rotation angle with a numeric value.
    \end{itemize}
    
    \item Algorithm:
    \begin{itemize}
        \item Extract the image dimension
        \item The \lstinline{cv2.getRotationMatrix2D()} function is used to compute the rotation matrix for the given angle around the center of the image.
        \item Apply the rotation defined by the rotation matrix \lstinline{rotation_matrix} to the original image using the function \lstinline{cv2.warpAffine()}. The argument \lstinline{(width, height)} specifies the size of the output image.

    \end{itemize}
\end{itemize}


\subsection{Output}
\begin{itemize}
    \item Return the rotated image:
\end{itemize}

\begin{figure}[H]
    \centering
    \includegraphics[width=0.75\textwidth]{lena_gray_rotated.jpg}
    \caption{Rotated Grayscale Image}
    \label{fig:rotated_grayscale_img}
\end{figure}

\printbibliography
\end{document}
